\documentclass{report}

\usepackage[utf8]{inputenc}

\usepackage[margin=0.75in]{geometry}
\usepackage[titletoc,title]{appendix}

% Odkomentuj by zmienić tytuł abstraktu
% \renewcommand{\abstractname}{Podsumowanie}

% Wcięcie pierwszego paragrafu po rozpoczęciu sekcji, inaczej wygląda dziwnie tbh
\usepackage{indentfirst}

% Matematyka
% https://www.overleaf.com/learn/latex/Mathematical_expressions
% https://en.wikibooks.org/wiki/LaTeX/Mathematics
\usepackage{amsmath,amsfonts,amssymb,mathtools,amsthm,empheq,mathabx}

\usepackage{polski}
\numberwithin{equation}{section}

\usepackage[figurename=Fig.]{caption}
\usepackage{fancyhdr}
\usepackage{enumitem}
\usepackage[ocgcolorlinks]{hyperref}
\hypersetup{
    ocgcolorlinks=true,
    linkcolor=blue!50!red,
    urlcolor=blue!70!black
}

% Obrazki i wykresy
% https://www.overleaf.com/learn/latex/Inserting_Images
% https://en.wikibooks.org/wiki/LaTeX/Floats,_Figures_and_Captions
\usepackage{tikz}
\usepackage{graphicx,float,pgfplots,float}
\pgfplotsset{compat=newest}
\usetikzlibrary{decorations.markings}
\graphicspath{ {./images/} }
% Algorytmy
% https://www.overleaf.com/learn/latex/algorithms
% https://en.wikibooks.org/wiki/LaTeX/Algorithms
\usepackage[ruled,vlined]{algorithm2e}
\usepackage{algorithmic}

% Wyświetlanie SVG, wymaga Inkscape'a
% \usepackage{svg}
% Syntax highlighting
% https://www.overleaf.com/learn/latex/Code_Highlighting_with_minted
\usepackage{minted}
\usemintedstyle{borland}
\renewcommand{\chaptername}{Zadanie}
\def\chapterautorefname{Zadanie}

\title{WCYB Projekt 2}
\author{Jakub Bliźniuk}
\date{December 2022}


\begin{document}
\pagestyle{fancy}
\fancyfoot{}
\fancyfoot[C]{\thepage}
\fancyfoot[R]{324904}

\begin{titlepage}
    \vfill
    \maketitle
    \thispagestyle{fancy}
\end{titlepage}
\tableofcontents
\thispagestyle{fancy}
\setcounter{chapter}{4}
\chapter{Serwer zdalny VPN}
\section{Wybrana technologia VPN}
Jako preferowaną technologię wybrałem Tailscale, z self-hostowanym serwerem kontrolnym - Headscale. Jest to w zasadzie technologia \textit{overlay networking}, pozwalająca na dynamiczne tworzenie połączeń między urządzeniami w sieci z użyciem Wireguard.

Konfiguracja była więc podzielona na trzy części:
\begin{enumerate}
    \item Konfiguracja serwera kontrolnego, który zarządza siecią, autoryzuje połączenia i uwierzytelnia węzły
    \item Konfiguracja węzła wychodzącego, w tym wypadku na tym samym urządzniu co serwer kontrolny
    \item Zalogowanie się do sieci w aplikacji Tailscale na urządzeniach końcowych
\end{enumerate}

Dodatkowo sieć została wykorzystana w kolejnym realizowanym zadaniu (\autoref{chap:pihole}), co wymagało tylko jednej zmiany w konfiguracji urządzeń wykorzystanych w tym zadaniu.

\section{Przygotowanie urządzeń}


\chapter{PiHole}
\label{chap:pihole}
\end{document}
